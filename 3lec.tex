\documentclass[12pt]{article}
\usepackage{blindtext}
\usepackage{hyperref}
\usepackage{amsmath}
\usepackage{fullpage}
\usepackage{amssymb}
 %\input chemfig.tex
\usepackage{chemfig}
\usepackage{background}
%\usepackage{float}
 
\usepackage{graphicx}
\graphicspath{{images/}}
\backgroundsetup{contents=structure dictates function,opacity=0.35,scale=4,color=blue,angle=40}
%\SetWatermarkText{STRUCTURE DICTATES FUNCTION}
\title{Summary of Lec 3}
\author{Milan Anand Raj\\manandraj20@iitk.ac.in}
\date{Jan 14}
%\usepackage{hyperref}
\begin{document}
\NoBgThispage
\maketitle



\begin{center}
\textbf{INDIAN INSTITUE OF TECHNOLOGY}

kanpur
\end{center}


\vfill
\begin{figure}
\centering
\includegraphics[scale=.1]{iitk.jpg}
\end{figure}
\clearpage
\tableofcontents
\clearpage
\NoBgThispage
\section{Kinase Enzyme}
Having been implicated in certain diseases including cancers,metabolic disorders and various central nervous system indications, this set of enzymes is potential candidate to get hands on!
About $1.7  $ percent of our genome code for kinases! This is gigantic considering the vast diversity of proteins that are coded.
We will see how it functions and what its structure is.



\subsection{Chain of operations }
They are tightly involved in cellular signalling networks.They are protein phosphorylates or,they add phosphate group to the hydroxyl terminal of proteins. They often need ATP for this purpose.
\begin{center}
\textbf{Extracellular signals like neurotransmitters, arrive }

$\downarrow$

\textbf{Intracellular impulse like cycle-GMP ,is generated.}

$\downarrow$

\textbf{Protein kinase wakes.}

$\downarrow$

\textbf{Substrate presence is detected and it is made to fit on the enzyme.}

$\downarrow$

\textbf{Physiological response is observed.}
\end{center}
Enzymes are also classified based on their specificity and the type of reactions they catalyze.



\subsection{Classification}
There are several grounds on which the kinases can be classified. However, we will be covering the most obvious functional classification.
\begin{enumerate}
\item Protein serine/Threonine kinases.
\item Protein Tyrosine kinase.

\item Dual specificity kinases-serine/Threonine kinases and Tyrosine kinases.
\item Broad specificity kinases and narrow specificity kinases.
\item Classified by activator 
\begin{itemize}
\item PKA
\item PKC
\item PKG
\end{itemize}
\end{enumerate}
\clearpage
\subsection{Structure of kinase}
It has two lobes.One is N-terminal lobe and another is C-terminal lobe.C-terminal lobe mainly includes $\alpha$-helices and N-terminal mostly include $\beta$-sheets. There is a hinge joining both the ends. This hinge is where adenosine moiety of ATP attach with bi-dentate H-bond.
Helix-C on the N-terminal side plays an important role in catalysis. There is a activation loop which spans both the lobes. The activation loop is required as kinases rest in incompetent state and it needs to be activated with certain regulatory entities.
\begin{figure}[h]
\centering
\includegraphics[scale=.5]{kinase.jpg}
\caption{Protein kinase}

\end{figure}
\clearpage
\subsection{Activation and conformational changes}
The arrival of Mg$^{2+}$ cofactor along with ATP causes the conformational changes needed for substrate to fit in. The two lobes slides against each other and make the substrate protein to fit in.

Receptor tyrosine kinases[Figure \ref{fig receptor}] are the membrane bound enzyme having both ligand binding domain and site for catalysis on the same polypeptide.Extracellular ligands causes the appropriate structural integration of the enzyme to carry out tyrosine phosphorylation.
\begin{figure}[h]
\centering
\includegraphics[scale=.5]{receptor.jpg}
\caption{Receptor kinase}
\label{fig receptor}
\end{figure}
There are various modes by which conformational changes in enzymes can be induced. Some of them include removal of regulatory subunit carrying pseudosubstrate, clearing the catalytic site blocked by an intrinsic domain carrying a pseudosubstrate, bringing the enzyme to correct conformation by phosphorylation or by attaching cyclic-regulators.

\end{document}