\documentclass[12pt]{article}
\usepackage{blindtext}
\usepackage{hyperref}
\usepackage{amsmath}
\usepackage{fullpage}
\usepackage{amssymb}
 %\input chemfig.tex
\usepackage{chemfig}
\usepackage{background}
\usepackage{float}
 
\usepackage{graphicx}
\graphicspath{{images/}}
\backgroundsetup{contents=structure dictates function,opacity=0.35,scale=4,color=blue,angle=40}
%\SetWatermarkText{STRUCTURE DICTATES FUNCTION}
\title{Summary of Lec 9}
\author{Milan Anand Raj\\manandraj20@iitk.ac.in}
\date{Jan 29}
\begin{document}
\NoBgThispage
\maketitle



\begin{center}
\textbf{INDIAN INSTITUE OF TECHNOLOGY}

kanpur
\end{center}


\vfill
\begin{figure}
\centering
\includegraphics[scale=.1]{iitk.jpg}
\end{figure}
\clearpage
\tableofcontents
\clearpage
\NoBgThispage
\section{Haemoglobin}
Haemoglobin is a spherical globular protein with 5.5 $nm$ in diameter. It is specialised in transporting oxygen to the tissues. It is also a globin protein. Last class we saw that myoglobin has such a high oxygen binding affinity. Then why is there a need for extra headache to handle in the form of Hb! The answer lies in the fact that we not only need high oxygen affinity but also sufficient oxygen delivery to peripheral tissues.

Hb has a special tendency to transition between two configuration namely T and R. These two configuration helps Hb to vary its oxygen binding affinity at will. Hb adopts R configuration near the lungs where oxygen needs to be taken up in the blood. On the other hand, they need to switch to T state near the tissues to unload the oxygen.

 

\subsection{Structure and Interaction }
Hb is a regular globin protein. It is tetrameric in nature with two $\alpha$ and $\beta$ subunits each. $\alpha$ subunits lack D helix. Each monomer incorporates a Heme group. Though there are not much similarity of peptide sequence between the two units, their structures are nearly superimposable.

The non-covalent interaction among the $\alpha1-\beta1$ and $\alpha2-\beta2$ protomers outnumber the interactions between $\alpha1-\beta2$ and $\alpha2-\beta1$ interfaces. Binding of oxygen only slightly change protomeric interactions. But at the $\alpha1-\beta2$ and $\alpha2-\beta1$ many ionic interactions are disturbed due to protein "breathing".

In T state, the Heme ring is slightly puckered and the $Fe^{2+}$ protrudes towards the proximal histidine. Upon oxygen binding this protrusion is inhibited which causes the conformational changes in the structure of that monomer. This structure change is transmitted to the $\alpha1-\beta2$ and $\alpha2-\beta1$ interfaces causing other subunits to opt for R state.
\subsection{Structural change in Hb on $O_{2}$ binding}
$O_{2}$ is the homotropic modulator or, it cooperatively attach to the binding site. The structural changes upon oxygen binding only triggers more oxygen binding to other monomeric subunits.There is a configurational transition i.e., $T\rightarrow R$. There is a major His residue and Lys residue ionic interaction at the $\alpha1-\beta2$ interface which is altered upon oxygen binding. This causes shortening of gap between beta subunits. The His residue, in the R state, turns towards the center of the molecule where they are no longer involved in ionic interaction to $\alpha$ unit.

When a ligand binds, the moving parts of the protein’s binding site may be stabilized in a particular conformation, affecting the conformation of adjacent polypeptide subunits.

\end{document}