\documentclass[12pt]{article}
\usepackage{blindtext}
\usepackage{hyperref}
\usepackage{background}
\usepackage{graphicx}
\graphicspath{{images/}}

\backgroundsetup{contents=structure dictates function,opacity=0.35,scale=4,color=blue,angle=40}
%\SetWatermarkText{STRUCTURE DICTATES FUNCTION}
\title{Summary of Lec 1}
%\usepackage{hyperref}
\author{Milan Anand Raj\\manandraj20@iitk.ac.in}

\date{Jan 14}
\begin{document}
\NoBgThispage
\maketitle
\begin{center}
\textbf{INDIAN INSTITUE OF TECHNOLOGY}

kanpur
\end{center}


\vfill
\begin{figure}
\centering
\includegraphics[scale=.1]{iitk.jpg}
\end{figure}
\newpage
\tableofcontents
\clearpage

\section{Enzymes}
Enzymes are biological agents without which our cellular activities would have taken years to complete! Even our most complex synthetic catalyst can't stand parallel to the structurally least convoluted enzyme!

The above two lines suffice to say why we should ponder on what enzyme is and how it functions.
\subsection{Properties of enzyme}
\begin{itemize}
\item Enzymes are biological catalysts which enhance the rate of biological reactions by figure of thousands and even more.
\item They are mostly proteins except for some of the catalytically active RNA{*} molecules. 
\item They are specific in nature.The complementarity of the substrate to active site of enzymes is an important factor.
\item They don't get consumed in a chemical reaction.
\item They work by lowering the activation energy of the reactions.
\item Their catalytic activity is greatly influenced by the integrity of their structure.
\item They work in a narrow range of temperature and pH.

\end{itemize}
\vfill
*RNA molecules are believed to have worked both as repositories of genetic information and as biological catalysts. Later ,DNA evolved as the most stable moiety to store genetic information and proteins as the enzymes.
\clearpage

\subsection{Structure of enzymes}
Their primary structure consist of long chains of covalently bonded amino acids. Their secondary structure builds on long polypeptides and is determined by the way differently positioned amino acids interact non-covalently.The most observed secondary structures are $\alpha$-helix and $\beta$ sheets. Tertiary structure is simply the coming together of different secondary structures in $3-D$ space.
Finally, two or more polypeptides mechanically interlock against each other to produce the final functioning quaternary structure.

\subsection{Non-covalent interactions}
\begin{itemize}
\item Hydrogen-bonds-A polar interaction between hydrogen and a polarised anion.
\item Di-sulphide bonds-VandeerWaal Interaction of two sulphur atoms.
\item Ionic bonds-Field interaction of a positive and a negative ion.
\item Hydrophobic interactions - VanDeerWal interaction of sterically hindered bulky groups.
\end{itemize}
\newpage
%\NoBgThisPage
\subsection{Working model}
Enzymes work by either rigid model or flexible model.

 In rigid model, it is assumed that enzyme has a definite convolution which substrate has to satisfy. On the contrary, flexible model provide us with feasible assumptions. It supports the fact that enzymes initially can be in any incompetent state and later ,upon trigger of its activation it changes to catalytically compatible configuration.

The commonality between the two is that the substrate need to be complementary in structure to pockets on the enzymes. Refer to [Figure \ref{fig key lock}]
\vfill
\begin{figure}[h]
\centering
\includegraphics[scale=.31]{keylock.jpg}
\caption{Key-Lock Mechanism}
\label{fig key lock}

\end{figure}


\end{document}