\documentclass[12pt]{article}
\usepackage{blindtext}
\usepackage{hyperref}
\usepackage{amsmath}
\usepackage{fullpage}
\usepackage{amssymb}
 %\input chemfig.tex
\usepackage{chemfig}
\usepackage{background}
\usepackage{float}
 
\usepackage{graphicx}
\graphicspath{{images/}}
\backgroundsetup{contents=structure dictates function,opacity=0.35,scale=4,color=blue,angle=40}
%\SetWatermarkText{STRUCTURE DICTATES FUNCTION}
\title{Summary of Lec 11}
\author{Milan Anand Raj\\manandraj20@iitk.ac.in}
\date{Feb 5}
\begin{document}
\NoBgThispage
\maketitle



\begin{center}
\textbf{INDIAN INSTITUE OF TECHNOLOGY}

kanpur
\end{center}


\vfill
\begin{figure}
\centering
\includegraphics[scale=.1]{iitk.jpg}
\end{figure}
\clearpage
\tableofcontents
\clearpage
\NoBgThispage
\section{Carbon monoxide poisoning}
$CO$ accounts for more than half of the poisoning worldwide! It is, therefore, crucial that we look at the way $CO$ operates with our blood.It turns out that their binding positively enhances  the association constant($K_{a}$) of $O_{2}$. Though more $O_{2}$ molecules are picked up at the lungs, they are not unloaded at the capillary tissues sufficiently. Fetal Hb (HbF) has been corroborated to have higher affinity for $CO$ both \textit{a priori and a posteriori}. Therefore, the fetus is highly vulnerable of $CO$ poisoning.
\vfill
\section{Sickle Cell Anaemia}
Because of a single point mutation in the $\beta$-Hb gene, Hb level of an individual is greatly attenuated. The Sickle cell Hb doesn't cause problems when it is loaded with oxygen. As it passes through the capillaries and $O_{2}$ molecules are unloaded, the sticky valine (substitution for Glutamine in the mutation) cause them to form fibriller aggregates before the cell manage to escape the capillaries. This induces sickling as Sickled RBCs interfere with the lining of capillary and membrane of peripheral tissues.

The mutation is not major a problem until and unless it is homozygous. The heterozygous Hb individuals don't display much of symptoms lest extreme physical exertion. The heterozygous mutation is generally seen in the african population and it has been deciphered that the heterozygous mutation help them beat \textit{malaria} better.

\end{document}