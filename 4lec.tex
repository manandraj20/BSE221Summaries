\documentclass[12pt]{article}
\usepackage{blindtext}
\usepackage{hyperref}
\usepackage{amsmath}
\usepackage{fullpage}
\usepackage{amssymb}
 %\input chemfig.tex
\usepackage{chemfig}
\usepackage{background}
%\usepackage{float}
 
\usepackage{graphicx}
\graphicspath{{images/}}
\backgroundsetup{contents=structure dictates function,opacity=0.35,scale=4,color=blue,angle=40}
%\SetWatermarkText{STRUCTURE DICTATES FUNCTION}
\title{Summary of Lec 4}
\author{Milan Anand Raj\\manandraj20@iitk.ac.in}
\date{Jan 14}
\begin{document}
\NoBgThispage
\maketitle



\begin{center}
\textbf{INDIAN INSTITUE OF TECHNOLOGY}

kanpur
\end{center}


\vfill
\begin{figure}
\centering
\includegraphics[scale=.1]{iitk.jpg}
\end{figure}
\clearpage
\tableofcontents
\clearpage
\NoBgThispage
\section{Proteolytic enzymes}
This class of enzymes is often involved in digestion of proteins into simplest forms ,degradation of certain pathogen proteins,curdling milk!
The other sort of work include helping sperm penetrate the ovum during fertilisation!
Often scientists have tried to encounter the viral attack by targeting the proteases of virus! In case of virus, it is the protease only which give the proteins their working form by cleaving the bonds at specific positions.

Let's have some experience with it.



\subsection{Classification }
Proteases are the proteins capable of catalyzing the hydrolysis of a protein substrate.

Based on their preference of cleavage,they are classified into two categories.
\begin{itemize}
\item Endopeptidase-They are the peptides that cleave in the interior region of the polypeptide chain.
\begin{itemize}
\item Serine proteases-Here, the hydroxyl group in the side chain of a serine residue in the active site acts as a nucleophile in the reaction that hydolyzes a peptide bond.
\item Cysteine proteases-The sulphydryl group of the side chain acts as a nucleophile.
\item Aspartic acid proteases-The water molecule in the active site functions as the nucleophile that attacks the peptide bond.
\item Metalloprotease-Again,the water molecule does the job.
\end{itemize}
\item Exopeptidase-They act at the end of peptide chain.
\begin{itemize}
\item Amino peptidase-They are the ones which cleave the peptide bond at the N-terminus.
\item Carboxy peptidase-They cleave at the C-terminus.
\end{itemize}
\end{itemize}


\section{Case study of $M^{pro}$}
$M^{pro}$ is a viral encoded cysteine protease. It has got a preference for substrate with glutamine residue at P1 site.

The $M^{pro}$ inhibitors, thus ,designed have residues mimicking the glutamine residue in the substrate.
\section{Discovery of enzyme inhibitors}
\begin{enumerate}
\item High throughput screening - In a loop, random substrate is made to settle with the enzyme and the enzyme activity is checked. The pseudosubstrate which mimics the substrate the best is then selected. It is a very inefficient process.
\item Structural-based design-We have a better way where based on 3-D structure of enzyme, we try to predict the substrate which will mimic the original substrate the most. This is then verified in vitro.

\end{enumerate} 
\end{document}