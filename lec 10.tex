\documentclass[12pt]{article}
\usepackage{blindtext}
\usepackage{hyperref}
\usepackage{amsmath}
\usepackage{fullpage}
\usepackage{amssymb}
 %\input chemfig.tex
\usepackage{chemfig}
\usepackage{background}
\usepackage{float}
 
\usepackage{graphicx}
\graphicspath{{images/}}
\backgroundsetup{contents=structure dictates function,opacity=0.35,scale=4,color=blue,angle=40}
%\SetWatermarkText{STRUCTURE DICTATES FUNCTION}
\title{Summary of Lec 10}
\author{Milan Anand Raj\\manandraj20@iitk.ac.in}
\date{Jan 29}
\begin{document}
\NoBgThispage
\maketitle



\begin{center}
\textbf{INDIAN INSTITUE OF TECHNOLOGY}

kanpur
\end{center}


\vfill
\begin{figure}
\centering
\includegraphics[scale=.1]{iitk.jpg}
\end{figure}
\clearpage
\tableofcontents
\clearpage
\NoBgThispage
\section{Model for cooperative binding of oxygen to Heme group}
$CO_{2}$
There are two proposed model to explain the cooperative binding of oxygen. One is concerted model and the other is sequential model.
\subsection{concerted model}
It assumes that all the subunits, at a time, has to be in one configuration either T or R.
\subsection{Sequential model}
It rather gives freedom to every subunit be in any configuration at different times. The models are not mutually exclusive. Rather concerted model is a subset of sequential model.


\section{Bohr effect}
The binding affinity of $O_{2}$ is compromised in the presence of $CO_{2}$. This happens at the capillary level where cells are breathing out $CO_{2}$, which increases the partial pressure of $CO_{2}$ in the extracellular matrix. This induces the transition of Haemoglobin from R to T state where $O_{2}$ binds with less affinity. Haemoglobins are also involved in transportation of $40 \%~H^{+}$ and 15 -20 $\% ~ CO_{2}$ to the lungs. Rest of $CO_{2}$ is transported through blood as $HCO_{3}^{-}$.
 
 The equation governing this $CO_{2}$ + $H_{2}O~ \rightleftarrows ~ HCO_{3}^{-}$+ $H^{+}$.
This reaction also drives the T $\rightarrow$ R transition as affinity of $O_{2}$ decreases with decrease in pH.
 This effect of pH and $CO_{2}$ on the binding affinity of $O_{2}$ is what is called Bohr effect.
 \clearpage
 \section{BPG regulation of $O_{2}$ binding to Hb}
Suppose you go to higher altitude. With increase in altitude the partial pressure of oxygen starts to fall down. This affects the oxygen \textbf{saturation} point of Hb in the lungs. The partial pressure of oxygen is maintained at 4 $ pKa$ at the tissues because of \textit{Homeostasis}. This causes the $\%$ oxygen delivery to fall from 38 $\%$ to 30 $\%$. This has great repercussions as our body often seems to work at narrow range of variation.

To cope up with this situation, cells start synthesising BPG which is a negative heterotropic modulator for Hb. This alters the oxygen binding to Hb negatively. The $O_{2}$ molecules are now heavily unloaded at the capillary level causing sufficient supply of oxygen to the tissues. 
To supplement the information, refer to [Figure \ref{fig BPG}]
\begin{figure}[h]
\centering
\includegraphics[scale=1.5]{BPG.png}
\caption{Regulation by BPG modulator}
\label{fig BPG}
\end{figure}
\end{document}