\documentclass[12pt]{article}
\usepackage{blindtext}
\usepackage{hyperref}
\usepackage{amsmath}
\usepackage{fullpage}
\usepackage{amssymb}
 %\input chemfig.tex
\usepackage{chemfig}
\usepackage{background}
\usepackage{float}
 
\usepackage{graphicx}
\graphicspath{{images/}}
\backgroundsetup{contents=structure dictates function,opacity=0.35,scale=4,color=blue,angle=40}
%\SetWatermarkText{STRUCTURE DICTATES FUNCTION}
\title{Summary of Lec 13}
\author{Milan Anand Raj\\manandraj20@iitk.ac.in}
\date{Feb 5}
\begin{document}
\NoBgThispage
\maketitle



\begin{center}
\textbf{INDIAN INSTITUE OF TECHNOLOGY}

kanpur
\end{center}


\vfill
\begin{figure}
\centering
\includegraphics[scale=.1]{iitk.jpg}
\end{figure}
\clearpage
\tableofcontents
\clearpage
\NoBgThispage
\section{Payoff Phase}
In the first step of Pay-off phase, GAP is made to react with an inorganic phosphate in the presence of a \textit{dehydrogenase}. The NAD$^{+}$ is reduced to NADH as GAP is oxidised. The seventh step of Glycolysis is where first ATP molecule is generated. Bis-phosphate is converted to monophosphate in the presence of \textit{Phosphoglycerate Kinase}. \textit{Phosphoglycerate Mutase} catalyze migration of Phosphate from one position to another in 3-phosphoglycerate in the 8$th$ step. Before the final step, dehydration of 2-phosphoglycerate is carried out to supplement the ease of ultimate step. Product concentration is kept low to pull the reaction forward. In the ultimate step, loss of phosphate from PEP to ADP yields an enol that tautomerises into ketone which drives the reaction forward by lowering the product conc.

2 ATP is harvested per molecule of GAP and hence 4/\textit{glucose}.

  
\section{Insulin Signalling}
Glucose uptake is mediated by a special hormone secreted in \textit{Pancreas}. Whenever there is a stretch in digestive tract or increase in glucose level in the blood, CNS commands \textit{pancreas } to secrete \textbf{insulin} into the blood which if detected help cells to initiate synthesis of glucose transporters and then integrated into the cell membrane by exocytosis. The feedback mechanism (lowered glucose level in the blood)  helps pancreas stop further secretion of insulin.

Two type of maladies are seen when there is a fault in the above circuit. First, Insulin is not synthesised in the pancreas. Second where cells don't respond to insulin properly.This increases the level of glucose in the blood. Increased level of glucose in the blood has severe repercussions. It damage the vessels and create plaques which increase the risk of heart attack, stroke and what not.




\end{document}