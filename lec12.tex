\documentclass[12pt]{article}
\usepackage{blindtext}
\usepackage{hyperref}
\usepackage{amsmath}
\usepackage{fullpage}
\usepackage{amssymb}
 %\input chemfig.tex
\usepackage{chemfig}
\usepackage{background}
\usepackage{float}
 
\usepackage{graphicx}
\graphicspath{{images/}}
\backgroundsetup{contents=structure dictates function,opacity=0.35,scale=4,color=blue,angle=40}
%\SetWatermarkText{STRUCTURE DICTATES FUNCTION}
\title{Summary of Lec 12}
\author{Milan Anand Raj\\manandraj20@iitk.ac.in}
\date{Feb 5}
\begin{document}
\NoBgThispage
\maketitle



\begin{center}
\textbf{INDIAN INSTITUE OF TECHNOLOGY}

kanpur
\end{center}


\vfill
\begin{figure}
\centering
\includegraphics[scale=.1]{iitk.jpg}
\end{figure}
\clearpage
\tableofcontents
\clearpage
\NoBgThispage
\section{Glucose}
Glucose, as we know, is the primary source of energy for living organisms for it yields high amount of energy upon oxidation, it is a precursor for most of our cellular biomolecules. Because it can be stored in polymeric form, our body stores it for future needs. Apart from generating \textbf{NADPH}, it also helps in synthesis of other carbohydrates which constitute the structural components of our cell.
\subsection{Glycolysis}
Though there are various pathways energy can be harvested from glucose, Glycolysis is the earliest and efficient one. Long before photosynthesis came to play, living creatures used to extract energy of glucose anaerobically. It supplements the synthesis of ATP and NADPH. The two hallmark of Glycolysis :-
\begin{itemize}
\item First: Activate Glucose by phosphorylation.
\item Second: Collect energy from the high-energy metabolites.
\end{itemize}

Glycolysis follow 10 steps, half of which constitutes \textbf{Preparatory phase} and another half \textbf{Payoff phase}.

\subsubsection{Preparatory Phase}
Glucose is first phosphorylated by hexokinase to yield activated glucose. This happens irreversibly in the presence of  ATP and $Mg^{++}$ (which mediates reaction by distracting -ve charge on ATP) by nucleophilic attack of C6 oxygen. The Glucose-6-phosphate converted so far is pushed to Fructose-6-phosphate in the presence of an isomerase. Though the reaction is thermodynamically not favourable, it is a pre-requisite for further steps.

\textbf{Note}: The reactions which are not thermodynamically favourable is made to proceed via lowered product concentration.

In the next thermodynamically favourable irreversible step, fructose is further phosphorylated by consuming an extra ATP. Fructose-bis-phosphate is cleaved by \textbf{aldose} into \textit{Glyceraldehyde-3-phosphate} and \textit{Dihydroxyacetone phosphate}. In order to generate two units of GAP to input the payoff phase, DHAP is isomerised to GAP.



\end{document}