\documentclass[12pt]{article}
\usepackage{blindtext}
\usepackage{hyperref}
\usepackage{amsmath}
\usepackage{fullpage}
\usepackage{amssymb}
 %\input chemfig.tex
\usepackage{chemfig}
\usepackage{background}
%\usepackage{float}
 
\usepackage{graphicx}
\graphicspath{{images/}}
\backgroundsetup{contents=structure dictates function,opacity=0.35,scale=4,color=blue,angle=40}
%\SetWatermarkText{STRUCTURE DICTATES FUNCTION}
\title{Summary of Lec 2}
\author{Milan Anand Raj\\manandraj20@iitk.ac.in}
\date{Jan 14}
\begin{document}
\NoBgThispage
\maketitle



\begin{center}
\textbf{INDIAN INSTITUE OF TECHNOLOGY}

kanpur
\end{center}


\vfill
\begin{figure}
\centering
\includegraphics[scale=.1]{iitk.jpg}
\end{figure}
\clearpage
\tableofcontents
\clearpage
\NoBgThispage
\section{Enzymes}
With mere presence ,enzymes have been found to lower the activation energy of biochemical reactions by great factors[Figure \ref{fig oip}]!Also there are certain convolutions on the surface of enzymes that can help target the enzymes!
\begin{figure}[h]
\centering
\includegraphics[scale=.6]{OIP.jpg}
\caption{Lowering of activation energy}
\label{fig oip}
\end{figure}
Making use of the allosteric sites, various drugs have been designed. Diving into the cryptic details of their mechanisms of operations is of great importance.
\subsection{Classification}
Though there are various grounds on which they can be classified. We will be looking at the most obvious of them.
\begin{enumerate}
\item Simply proteins.
\item Holoenzymes-They work in coordination with some other chemical moiety.The protein part of holoenzyme is called apoenzyme. And, the non-protein part is called cofactor.

Cofactors branch to two sections.
\begin{itemize}
\item Prosthetic group-They are small inorganic molecules tightly bonded to enzymes.For example,$Fe^{2+}$ in haemoglobin.
\item Coenzyme-They are large organic molecules which loosely bind to enzymes. For example, vitamins.
\end{itemize}
\end{enumerate}
\clearpage
\subsection{Michaelis-Menten Equation}
Based on some hypothetical assumptions ,Michalis-Menten reached to a relation between rate of reaction and the conc. of substrate at any point in time.
\begin{center}

$$v_{o}=\frac{V_{max}[S]}{K_{m}[S]}$$
\end{center}
where
\begin{align}
v_{o}&=the ~initial~ rate~ of~ reaction\\
[S] &= {conc.}~ of ~substrate ~at~ any ~point~ in~ time\\
V_{max}&=the~ maximum~ reaction~ rate~ that~ can~ be~ achieved ~at ~the ~given ~enzyme~ conc.\\
K_{m}&=conc.~of~ substrate~ when~ reaction~ rate~ is~ \frac{V_{max}}{2}.
\end{align}
\begin{figure}[h]
\centering
\includegraphics[scale=.4]{maxresdefault.jpg}
\label{fig mme}
\caption{Michaelis-Menten Equation}

\end{figure}

\subsubsection{Deciphering the equation}
\begin{enumerate}
\item The substrate concentration has a profound effect on the rate of enzyme catalyzed reactons.

\schemestart \chemname{E}{Enzyme}\hspace{1cm}     +  \hspace{1cm}      \chemname{S}{substrate}\arrow{<=>[$k_{1}$][$k_{-1}$]}[30,1.5]\hspace{1cm}\chemname{E-S}{Enzyme substrate complex}\hspace{1cm}\arrow{<=>[$k_{2}$][$k_{-2}$]}[-30,1.5]\chemname{E}{enzyme}\hspace{1cm}+\hspace{1cm}\chemname{P}{product}\schemestop

\item The rate of the catalyzed reaction will obviously be at a maximum when virtually all of the enzyme is present in the ${ES}$ complex and the conc. of free enzyme is vanishingly small.
\item Since the second reaction is slow rate determining step, the enzyme will always be saturated of the substrate in excess.

\end{enumerate}
\clearpage
\subsection{Inhibition of enzymes}
The inhibition of enzyme function can happen in two ways.
\begin{itemize}
\item Reversible Inhibition - Pseudo substrate imitating the structure of original substrate. It can be overcome by having original substrate in excess.
\item Irreversible Inhibition - Some regulatory moiety attaches to the allosteric site on the enzyme bringing in the conformational changes in the structure of enzyme. Substrate conc. has no effect in reversing the situation.
\end{itemize}

\subsection{Transition state complementarity of enzyme}
Having an enzyme that has structure complementary to the structure of transition state, we can reduce the activation energy by a great factor. This is accompanied because of energy favourable interaction between enzyme and transition state. On the other hand, complementarity to substrate merely diverts the reaction. 

\end{document}